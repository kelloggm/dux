\title{Dux: Automating Dependency Management -- Windows Porting}
\author{
        Omeed Magness \\
        University of Washington
            \and
        Martin Kellogg \\
        University of Washington
}

\documentclass[10pt,conference]{IEEEtran}

\usepackage{graphicx}
\graphicspath{ {images/} }
\usepackage{myref}
\usepackage{booktabs}
\usepackage{url}
\usepackage{listings}
\usepackage{color}
\usepackage{tikz}

\begin{document}

\maketitle

\lstset{language=bash}

\begin{abstract}

The Dux build orchestration tool~\cite{dux} provides a guarantee that
after observing a successful build of a project, it can reproduce this build on
any other machine by saving a configuration file and dependency artifacts
to the cloud. However, Dux was limited to Linux because of its dependency on strace,
Linux's system call tracing utility.

In this report, we discuss modifications to Dux that allow it to run on Windows. 
We evaluated and selected a suitable replacement for strace on Windows, rewrote
the tracer and parser to target this new tool, and rewrote the backend logic of
Dux to handle the new trace format.

In the process, we refactored and decoupled a cross-platform strace-like library
which provides a consistent interface to a system call tracing tool across
Windows and Linux.

\end{abstract}

\section{Introduction}

The Dux build orchestration tool~\cite{dux} traces a demonstrated build command
and observes the system calls that the build command makes. From this list of
system calls, it observes which files a build is accessing. Dux saves these files along
with any relevant environment variables, preserving the ``execution context'' that
made that build possible. Dux can then recreate the build on another computer. 

Although Dux itself is written in Java (making the majority of the code
automatically cross-platform), the tracing currently requires Linux's strace utility. 
We want Dux to be cross-platform, because the best tools that become
widely adopted in the community are always cross-platform (e.g. git, LaTeX, etc.).
To make Dux cross-platform, we must find a suitable replacement
for strace on other platforms. 

This paper provides two main contributions. First, we present a platform-independent
strace library which is decoupled from Dux and can be used in other projects.
Second, we use this library (as a client) within Dux to trace, parse, filter, and
construct objects representing the output of a system call tracing utility. Dux
makes general-purpose, platform-independent calls into the library to trace
the execution of a process and the library dispatches to the appropriate tool
for the operating system on which the library is running, then processes the
output appropriately and provides a platform-independent way to discover the
contents of the execution trace.

The remainder of this paper discusses porting Dux to Microsoft Windows. Section II
describes how we picked a replacement for strace on Windows. Section 
III discusses the implementation of the port. Section IV evaluates the end result
and discusses limitations and future work.

\section{Finding a Replacement for Strace} 

The first task in porting Dux to Windows was finding a suitable replacement
for strace. We evaluated four open-source command-line tools that claimed to be
similar to strace: DrMemory\footnote{\url{http://www.drmemory.org/}},
StraceNT\footnote{\url{http://intellectualheaven.com/default.asp?BH=StraceNT}},
the strace Cygwin port\footnote{\url{http://cygwin.com/cygwin-ug-net/strace.html}},
and NTTrace\footnote{\url{http://www.howzatt.demon.co.uk/NtTrace/}}. 
All had significant limitations -- they could only be used on GUIs, could only
be use on command-line applications, or were extremely unreliable and frequently
crashed (Fig. 1). All of these projects were old, and development effort had been
abandoned in recent years. Although each of these may have been excellent tools
when actively developed, they no longer did the job.

\begin{figure}
\begin{center}
\begin{tabular}{ |p{1.6cm}|p{1.5cm}|p{1 cm}|p{3 cm}|}
 \hline
 Tool & Can trace: & Reliable & Notes \\ \hline \hline
 Dr.Memory & GUIs, command-line & Yes & Works well on Windows 7, but not on Windows 10? \\ \hline
 StraceNt & GUIs (sort-of) & No & Crashes for GUIs and command-line. \\ \hline
Cygwin strace port & command-line & Yes & Seems to only be able to trace other Cygwin binaries. \\ \hline
NTTrace & GUIs, command-line & Sort-of & Has issues dealing with 32 vs. 64-bit tracing; frequently errors. \\
 \hline
\end{tabular}
\end{center}
\caption{Comparison of the primary strace replacements we evaluated on Windows.
We also briefly investigated Event Tracing for Windows (ETW), although this seemed
to require a lot more work (essentially a re-implementation of strace), so we
abandoned that approach.}
\end{figure}

For that reason, we chose Process 
Monitor\footnote{\url{https://docs.microsoft.com/en-us/sysinternals/downloads/procmon}},
Microsoft's official process tracing utility. Process Monitor is reliable, is
still actively developed, and produces very thorough event traces (system calls, 
registry accesses, etc.) for all processes running on a system. Unfortunately, it is not
exactly meant for use like strace -- most operations must go through GUIs and,
in general, it traces all events across all processes on the system at once, not just 
an individual process. Although its advanced GUI filtering makes it possible to
filter by process ID (PID), it is hard to trace only the events of an arbitrary process (one whose
PID is not known ahead of time).

So, although Process Monitor was not exactly the tool we wanted, we chose it
because of its reliability and data quality. We found a command-line option to
extract a CSV file of event data from Process Monitor, and then parsed that data.

\section{Implementation}

\subsection{Environment Setup}

The next step in porting Dux to Windows was setting up Dux's development environment,
which entailed installing Google's Bazel build system, configuring a Google Cloud 
bucket (for use as a Dux file server), configuring credentials and service account
keys for that account, and integrating all of this with the Travis CI tool. The setup
turned out to be nontrivial, and we updated the README appropriately so
that future developers joining Dux will have a much smoother time.

\subsection{Reworking the Strace Dependency}

Next, we decoupled the strace dependence in Dux (moving the tracing code to a
separate package) and refactored the existing strace code to be more generic. 
Ideally, if a wrapper can be created around the functionality of an strace-like tool
on each platform, it is possible to make a ``cross-platform strace tool library"
which could be used in projects other than Dux.

Then, we began creating the batch scripts and configurations necessary to
extract data from Process Monitor. Once we had this data, parsing it was 
fairly simple, since it was a CSV file. The existing Dux parser was refactored, with
an inheritance hierarchy to support parsing across multiple platforms.

\subsection{Filtering the Process Monitor Data}

With the Process Monitor data parsed out and represented as Java objects ("StraceCall"
objects), the data still needed to be filtered down. Because Process Monitor's filters
are fairly GUI-oriented, it was significantly easier to do the event filtering
after the data collection.

We generalized certain strace flags to work with Process Monitor as well, by using
the Builder pattern. For example, clients of our cross-platform strace-tool library
can now ask to trace subprocesses on a trace, and this will dispatch to \texttt{-f} on
Linux (the flag for strace to trace subprocesses) , or post-collection filtering
on Windows. Because the data is all mixed together across processes in Windows,
we use a fixed point algorithm to determine subprocess PIDs.

Using this strace tool library, we filtered down to system calls
only relevant to Dux (such as creating a file or process), and also
filtered by PID to only keep system calls originating from the desired process
to trace and its subprocesses. 

\subsection{Adjusting the Strace Call Interface}

Once we had a filtered set of StraceCall Java objects representing the Process
Monitor data, the next step was to use this data. We wanted to provide a generic,
cross-platform way for a client to extact properties about a sytem call, so we
created some wrapper functions for these StraceCall objects to determine if,
for example, the call was an exec, open, etc.

In other words, rather than hardcoding a call to strace, manually parsing, 
and inspecting properties of each StraceCall object, a client of our strace tool
library now just specifies some behavior they want from the trace
(e.g. tracing subprocess, or filtering to certain calls) and they will get back a list
of the traced calls, already parsed and filtered. And, to interact with these calls,
there is a platform independent interface. With this interface adjusted, the
backend of Dux could now be re-written to use the new interface. This essentially
completed an initial implementation of porting Dux to Windows!

\section{Evaluation, Limitations, and Future Work}

To evaluate the prototype of the port of Dux to Windows, we tested a toy
example. First, we installed a free Pascal compiler (fpc) 
\footnote{\url{https://www.freepascal.org/}} onto a Windows workstation,
and then compiled a ``Hello World" program. Dux saved a configuration 
file and the necesssary dependencies to the cloud. We then went to another Windows
workstation without fpc installed, and asked Dux to check the build file for missing
dependencies. It downloaded the fpc binaries, set certain environment variables,
and then went on to compile the ``Hello World" program. When run, that program
produced the expected output.

We also did some stress testing and saw that Dux could bootstrap itself (i.e. trace
and record its own Google Bazel build). This involved uploading many hundreds
of files to the cloud (and so was quite slow). We are still evaluating if it was
necessary to upload as many dependencies as we did. Some of these 
``dependencies'' may have actually just been user-specific
temporary files that should have been excluded. This behavior could be due
to the fact that the Process Monitor output format makes it tricky to handle file
system links. Investigating and addressing this issue is future work.

The Windows port of Dux suffers from many of the same limitations as the original
Dux project -- in particular, the static versioning. This project also has added
the limitation that a Dux build file for a project on Windows will likely
not be able to reconstruct the environment on Linux properly, and vice versa,
because of differing binaries. We see this port as useful for the sake of one
Windows developer demonstrating a build to another Windows developer, or, as 
with the original  version of Dux, one Linux developer demonstrating a build to 
another Linux devloper. It is reasonably common for co-developers to conform
to some standard environment. 

As an orthogonal point, our ``cross-platform strace-tool" library is truly meant to be
cross-platform. In future work, we would like to generalize the 
interface of this strace tool library so that it can be released as its own tool. 
As of now, its functionality is fairly specific to Dux's use case.

\section{References}

\begingroup
\renewcommand{\section}[2]{}%

% The following two commands are all you need in the initial runs of
% your .tex file to produce the bibliography for the citations in your
% paper.
\bibliographystyle{plain}
\bibliography{genprog-bib/merged}
% You must have a proper ``.bib'' file
% and remember to run:
% latex bibtex latex latex
% to resolve all references
%
% ACM needs 'a single self-contained file'!
%
\endgroup

\end{document}
