\title{Dux: Automating Dependency Management}
\author{
        Martin Kellogg \\
        University of Washington
            \and
        Steven Lyubomirsky\\
        University of Washington
}


\documentclass[10pt,conference]{IEEEtran}

\usepackage{graphicx}
\usepackage{myref}
\usepackage{booktabs}
\usepackage{url}
\usepackage{listings}
\usepackage{color}
\usepackage{tikz}

\begin{document}

\maketitle

\lstset{language=bash}

\begin{abstract}

Research into build systems has focused on compiling and testing code, resulting
in high-quality tools for these activities. However, an often-overlooked step in
building code is \textit{dependency management}---the process of setting up the
environment so that the build can succeed. We argue that this process should be
treated as a separate problem from compiling and testing---the traditional domain
of build systems. Instead, we propose a \textit{build orchestration tool}, called
Dux, which focuses on the dependency management problem and then delegates to a
traditional build system for compiling and testing. Dux observes a successful
build for a project, and writes the information necessary to reproduce the build
to a configuration file. Further, Dux uses a content-addressed distributed file
system as a backing store for dependencies. When a project is observed to require
a dependency that is not in Dux's file system, it uploads a copy of the dependency.
By reading the configuration file, Dux can reproduce the build of the project on
a new system by checking the system for the required dependencies, and downloading
any missing ones using the copy created when creating the configuration file,
trivially permitting easy, reproducible builds.

\end{abstract}

\section{Introduction}

Modern build systems typically combine two functions:
setting up the environment and ensuring dependencies are available (\textit{build orchestration})
and compiling and testing (\textit{building}).
Existing tools are very good at the latter, but often fail at the former;
as an example, a study at Google found that more than 50\% of build failures
by real developers were caused by dependency errors~\cite{googlebuilderrors}.
We propose that by separating build orchestration from the process of actually
compiling and building software we can improve the reliability and reproducibility of builds.
We propose a specialized build orchestration tool which is responsible for setting up the environment
and providing dependencies, and then dispatches to a traditional build system.
By separating the two tools, we can achieve several advantages over traditional tools:
\begin{itemize}
\item{
\textit{Language independence:}
A specialized build orchestration tool can operate on any language.
Anecdotally, we have noticed that the best available build tools specialize
in a particular language (e.g. \texttt{ocamlbuild} or \texttt{gradle}).
A language-independent build orchestration tool can allow projects to retain
their natural build system while simplifying inter-language dependencies.
}
\item{
\textit{Automation:}
Modern industrial build systems (like Google's Bazel~\cite{blaze} or Facebook's Buck~\cite{buck}) provide fast,
reproducible builds---but at the cost of large, brittle configuration files.
By decoupling building from dependency management, we can automatically
discover dependencies and write configuration files automatically.
By combining this feature with the ability to dispatch to existing build systems,
our proposed tool makes migrating an existing project a push-button task.
}
\end{itemize}

\section{Technical Approach}

\subsection{Migration}

Migrating a project to use Dux is simple.
The developer demonstrates a build to Dux by providing a command (e.g. \texttt{ant build-and-test}),
and Dux executes that command in a sandbox, which records information about what the build does:
which files it accesses, which environment variables it reads, which programs it executes, etc.
Dux uses Linux's \texttt{strace} utility to perform this analysis---though this limits Dux to
Linux-based platforms, similar tools exist on other platforms: \texttt{dtruss} for Mac OSX or
\texttt{Process Monitor} for Windows. Extending Dux to operate on these platforms is future work.

Dux records the information it captures about the build into a configuration file.
To ensure that the same versions of artifacts are used during future builds,
Dux uses a hashing function to obtain a hash of each artifact required by the build.
Dux then communicates with a remote content-addressed file system~\cite{venti},
which is indexed by the hashes and contains artifacts.
If the artifact in question is not already in this remote file system, the migration tool uploads it.
The hashes (keys into the remote file system) are recorded in the configuration file.
Once the configuration file has been generated, it can be committed to version control.

\subsection{Building}

If a configuration file created by Dux is present, Dux the system for the dependencies required,
and downloads any that are not present.
It sets up the environment exactly as the configuration file specifies.
Then, it dispatches to the underlying build system to complete the build---Dux is specialized
to dependency management, and does not know or reason about actually building the software.
By separating these concerns, Dux can provide reproducible builds even for legacy software
with significant customized build logic, because migrating to use Dux requires only
demonstrating a correct build.

\section{Project Timeline}

For each project milestone, we have identified our goals for Dux by that time. The tool described
in the previous sections is the final product.

\subsection{First Checkpoint}

\begin{itemize}
\item{Dux can observe a program and determine which \texttt{strace} events are relevant.}
\item{A content-addressable file system has been identified or written
that Dux will use to store depedency artifacts.}
\item{A configuration file format has been identified.}
\item{Dux can read a configuration file and print text that indicates what actions it
should perform based on the configuration file, though it may not actually perform the actions.}
\end{itemize}

\subsection{Second Checkpoint}

\begin{itemize}
\item{Dux can observe a build and all files opened and all programs executed.}
\item{Dux can send required dependencies to a central server, which uses a content-addressed filesystem to store them.}
\item{Dux can convert a build observation into a configuration file that allows the build to be
repeated on a machine without the required dependencies installed,
by reading the configuration file and downloading the dependencies from the
central server.}
\item{Dux provides an abstraction of an underlying build system that supports at least one build systems.}
\end{itemize}

\subsection{Final Project}

\begin{itemize}
\item{Dux can observe a build and all files opened and all programs executed.}
\item{Dux can distinguish relative and absolute paths.}
\item{Dux can tell the difference between files part of the project and files that are not
(i.e. the difference between build artifacts and dependencies’ artifacts).}
\item{Dux can send required dependencies to a central server, which uses a content-addressed filesystem to store them.}
\item{Dux can convert a build observation into a configuration file that allows the build to be
repeated on a machine without the required dependencies installed,
by reading the configuration file and downloading the dependencies from the
central server and putting them in the expected places.}
\item{Dux provides an abstraction of an underlying build system that supports at least two build systems.}
\end{itemize}

\subsection{Stretch Goals}

\begin{itemize}
\item{Dux includes a library that abstracts \texttt{strace}'s functionality, and is able to use a similar tool
on at least one other platform.}
\item{Dux can record the values of relevant environment variables in the configuration file,
and reason about paths or chains of environment variable reads to capture a complete state.}
\end{itemize}

\section{Related Work}

Several modern build systems include a dependency management strategy. Tools like
Maven~\cite{Maven}, Gradle~\cite{Gradle}, Bazel~\cite{blaze}, or Buck~\cite{buck}
are primarily systems for building and testing software, but each includes the
ability to specify dependencies manually in some kind of configuration file.
Unlike Dux's configuration files, which are automatically generated, the configuration
files for these tools use domain-specific languages that developers have to learn,
write, and maintain. Some tools (such as Bazel) also include sandboxing tools
that force the build to only use dependencies explicitly specified in the
configuration file.

Most recent advances in build technology have come from industry, not from
academia. Adams and McIntosh breifly cover build systems in their survey of
release engineering~\cite{adams2016modern}, in an effort to urge academics to
pay more attention to the work going on in industry. Cox et al. examined how
to prioritize updating dependencies, but did not attempt to fix build issues~\cite{cox2015measuring}.
McIntosh et al. built a classifier to determine when changes to the build
system would be necessary based on the changes introduced to source code~\cite{mcintosh2014mining}.
Seo et al. performed an empirical study at Google, and determined that more
than 50\% of build breakages were caused by dependency errors~\cite{googlebuilderrors}.
This is especially motivating for our work, since Google's Bazel build system
is considered the industry standard---and dependency issues are their most
common build error.

\section{Future Work}
A key question remains, even once Dux is implemented: how does Dux deal with
change, either in the environment or in the code it is building? We view building
the version of Dux we described here as the first step towards an adaptive
build system that can repair build scripts when they fail, using techniques
from the automated program repair literature. Existing program repair techniques
suffer from high false positive rates, in large part due to the unstructured
nature of the problem---with only a few (hundred? thousand?) test cases, it is
difficult to model all the correct behavior of a large C program~\cite{genprogisbadrinard}.
By contrast, the structured nature of a build configuration file and the relatively
small size of the change set make build script repair an attractive domain for
automated repair tools. By combining a tool like Dux with a build script repair
system, one can imagine a build tool that can correctly deploy software even in
the face of different dependency requirements or a different platform automatically,
making deploying software in new environments a much simpler process.

\section{References}

\begingroup
\renewcommand{\section}[2]{}%

% The following two commands are all you need in the initial runs of
% your .tex file to produce the bibliography for the citations in your
% paper.
\bibliographystyle{plain}
\bibliography{genprog-bib/merged}
% You must have a proper ``.bib'' file
% and remember to run:
% latex bibtex latex latex
% to resolve all references
%
% ACM needs 'a single self-contained file'!
%
\endgroup

\end{document}
